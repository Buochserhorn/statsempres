%%%%%%%%%%%%%%%%%%%%%%%%%%%%%%%%%%%%%%%%%
% Beamer Presentation
% LaTeX Template
% Version 1.0 (10/11/12)
%
% This template has been downloaded from:
% http://www.LaTeXTemplates.com
%
% License:
% CC BY-NC-SA 3.0 (http://creativecommons.org/licenses/by-nc-sa/3.0/)
%
%%%%%%%%%%%%%%%%%%%%%%%%%%%%%%%%%%%%%%%%%

%----------------------------------------------------------------------------------------
%	PACKAGES AND THEMES
%----------------------------------------------------------------------------------------

\documentclass{beamer}

\mode<presentation> {

% The Beamer class comes with a number of default slide themes
% which change the colors and layouts of slides. Below this is a list
% of all the themes, uncomment each in turn to see what they look like.

%\usetheme{default}
%\usetheme{AnnArbor}
%\usetheme{Antibes}
%\usetheme{Bergen}
%\usetheme{Berkeley}
%\usetheme{Berlin}
%\usetheme{Boadilla} %like
%\usetheme{CambridgeUS}
%\usetheme{Copenhagen}
%\usetheme{Darmstadt}
%\usetheme{Dresden}
%\usetheme{Frankfurt}
%\usetheme{Goettingen} %like
\usetheme{Hannover} %like
%\usetheme{Ilmenau}
%\usetheme{JuanLesPins}
%\usetheme{Luebeck}
%\usetheme{Madrid}
%\usetheme{Malmoe}
%\usetheme{Marburg}
%\usetheme{Montpellier}
%\usetheme{PaloAlto}
%\usetheme{Pittsburgh}
%\usetheme{Rochester}
%\usetheme{Singapore}
%\usetheme{Szeged}
%\usetheme{Warsaw}

% As well as themes, the Beamer class has a number of color themes
% for any slide theme. Uncomment each of these in turn to see how it
% changes the colors of your current slide theme.

%\usecolortheme{albatross}
%\usecolortheme{beaver}
%\usecolortheme{beetle}
%\usecolortheme{crane}
%\usecolortheme{dolphin}
%\usecolortheme{dove}
%\usecolortheme{fly}
%\usecolortheme{lily}
%\usecolortheme{orchid}
%\usecolortheme{rose}
%\usecolortheme{seagull}
%\usecolortheme{seahorse}
%\usecolortheme{whale}
%\usecolortheme{wolverine}

%\setbeamertemplate{footline} % To remove the footer line in all slides uncomment this line
%\setbeamertemplate{footline}[page number] % To replace the footer line in all slides with a simple slide count uncomment this line

%\setbeamertemplate{navigation symbols}{} % To remove the navigation symbols from the bottom of all slides uncomment this line
}

\usepackage{graphicx} % Allows including images
\usepackage{booktabs} % Allows the use of \toprule, \midrule and \bottomrule in tables
\usepackage{pgfpages}
\usepackage{amsmath}
\usepackage{pgfplots}
\usepackage{tikz}

%----------------------------------------------------------------------------------------
%	TITLE PAGE
%----------------------------------------------------------------------------------------

\title[Computation \& optimization]{Computation \& optimization for Lasso - part 2} % The short title appears at the bottom of every slide, the full title is only on the title page

\author{Luyang Han \& Janosch Ott} % Your name
\institute[] % Your institution as it will appear on the bottom of every slide, may be shorthand to save space
{
ETH Zürich \\ % Your institution for the title page
%\medskip
%\textit{john@smith.com} % Your email address
}
\date{22 October 2018} % Date, can be changed to a custom date

\setbeamercovered{transparent} % else hidden elements are gray, this way they are invisible
\setbeamertemplate{navigation symbols}{} %comment to have a lot of navigating symbols
\setbeamertemplate{section in toc}[sections numbered] % removes the ugly balls
%\setbeameroption{show notes}
\setbeameroption{show notes on second screen=right}
\setbeamertemplate{enumerate items}[default] % to get rid of some more ugly balls


%%%%%%%% ------------%%%%%%%%%%-----------%%%%%%%%%%%------------%%%%%%%%%%-----------
\newcommand{\R}{\mathbb{R}}
\newcommand{\Norm}[1]{\left\lVert#1\right\rVert}
\newcommand{\norm}[1]{\left\lvert#1\right\rvert}
\DeclareMathOperator*{\argmin}{arg\,min}
\DeclareMathOperator*{\argmax}{arg\,max}



%%%%%%%%%%% ----------%%%%%%%%%%-----------%%%%%%%%%%------------%%%%%%%%%--------------


\usepackage{hyperref}



\begin{document}

\begin{frame}
\titlepage % Print the title page as the first slide
\end{frame}

\begin{frame}
\frametitle{Overview} % Table of contents slide, comment this block out to remove it
\tableofcontents % Throughout your presentation, if you choose to use \section{} and \subsection{} commands, these will automatically be printed on this slide as an overview of your presentation
\end{frame}

%----------------------------------------------------------------------------------------
%	PRESENTATION SLIDES
%----------------------------------------------------------------------------------------

%------------------------------------------------
\section{Coordinate Descent}
%------------------------------------------------



%------------------------------------------------
\section{A Simulation Study}
%------------------------------------------------



%------------------------------------------------
\section{Least Angle Regression}
%------------------------------------------------




%------------------------------------------------
\section{Digression: Duality}
%------------------------------------------------

\begin{frame}
\frametitle{Digression: Duality in optimization}

%\begin{tabular}{lcc}
%\toprule
%&Primal&Dual\\\midrule
%Optimize&$\min f(x)$&$\max q(\lambda,\mu)$\\\midrule
%Constraints&$g_i(x)\le0, h_j(x)=0, x\in X$&$\lambda\ge0$\\\midrule
%Function&$L(x,\lambda,\mu):=f(x)+\sum_i\lambda_i g_i(x)+\sum_j\mu_j h_j(x)$&$q(\lambda,\mu)=\inf\limits_{x\in X} L(x,\lambda,\mu)$\\
%&$L(x,\lambda,\mu):=f(x)+\lambda^Tg(x)+\mu^T h(x)$&\\\midrule
%\end{tabular}



%\vspace{15pt}
%Why though? - \textbf{Dual problem is always convex!}

\end{frame}
\note{In various section, I came across terms like "dual" and "dual problem"}

\begin{frame}
\begin{tabular}{ll}
\toprule[1.5pt]
\multicolumn{2}{c}{Primal}\\\midrule
Optimize&$\min f(x)$\\\midrule
Constraints&$g_i(x)\le0, h_j(x)=0, x\in X$\\\midrule
Function&$L(x,\lambda,\mu):=f(x)+\sum_i\lambda_i g_i(x)+\sum_j\mu_j h_j(x)$\\\midrule
%&$L(x,\lambda,\mu):=f(x)+\lambda^Tg(x)+\mu^T h(x)$\\\midrule
\multicolumn{2}{c}{Dual}\\\midrule
Function&$q(\lambda,\mu)=\inf\limits_{x\in X} L(x,\lambda,\mu)$\\\midrule
Constraints&$\lambda\ge0$\\\midrule
Optimize&$\max q(\lambda,\mu)$\\\midrule
\end{tabular}
\vspace{15pt}

Why though? - \textbf{Dual problem is always convex!}

\end{frame}

\note{
$x\in X$ for e.g. solutions in a cone or integer solutions

%$g$ and $h$ are now in vector notation

Terms: Primal problem, Lagrange function with dual variables/Lagrange-multipliers, dual function, dual problem


Dual problem is always convex! - I don't know much about optimization yet, but
they really like convexity.

"(Convexity confers two advantages. The first is that, in a constrained problem, a convex feasible region makes it easier to ensure that you do not generate infeasible solutions while searching for an optimum.)

The second advantage is that all local optima are global optima. That allows local search algorithms to guarantee optimal solutions. And local search is often faster." \cite{ru16})



}



%------------------------------------------------
\section{ADMM}
%------------------------------------------------

\begin{frame}
\frametitle{Alternating Direction Method of Multipliers (ADMM)}
Problem\only<2->{ - decomposable ! }
\only<2>{\note{decomposable problem and constraints!}
}
\[\underset{\beta\in\R^m, \theta\in\R^n}{\text{minimize}}f(\beta)+g(\theta)\quad\text{subject to}\ \mathbf{A}\beta+\mathbf{B}\theta-c=0\]
%\note{decomposable problem and constraints!\\}
Lagrangian\only<3->{ - decomposable !}
\only<3>{\note{Lagrangian problem can still be decomposed into $\beta$ and $\mu$ terms

this has nice algorithm where we can execute some stuff in parallel, because we can decompose the Lagrangian}
}
\[f(\beta)+g(\theta)+\left\langle\mu,\mathbf{A}\beta+\mathbf{B}\theta-c\right\rangle\]
Augmented Lagrangian\only<4->{ - NOT decomposable !}
\only<4>{
\note{
Augmented: scalar product with $\rho$ gets added, 

Method of Multipliers: is a way to make the algorithm more robust 

advantage: better convergence

disadvantage: no longer parallel execution of subtasks due to l2-term, no longer decomposable in beta and theta terms, as l2 norm dquares every entry of the vector

alternating direction: semi-decomposable, i.e. keeping one variable fixed while updating the other

$\rho$ is step length of iterative algorithm

All notes on this slide: see the slides by \cite{boyd}

}
}
\[L_{\rho}(\beta,\theta,\mu):=f(\beta)+g(\theta)+\left\langle\mu,\mathbf{A}\beta+\mathbf{B}\theta-c\right\rangle+\frac{\rho}{2}||\mathbf{A}\beta+\mathbf{B}\theta-c||_2^2\]
\end{frame}



\begin{frame}
\frametitle{Dual Variable Update}
\framesubtitle{Alternating Direction Method of Multipliers}
\begin{align*}
\beta^{t+1}&=\argmin_{\beta\in\R^m}L_{\rho}(\beta,\theta^t,\mu^t)\\
\theta^{t+1}&=\argmin_{\theta\in\R^m}L_{\rho}(\beta^{t+1},\theta,\mu^t)\\
\mu^{t+1}&=\mu^t+\rho(\mathbf{A}\beta^{t+1}+\mathbf{B}\theta^{t+1}-c)
\end{align*}

\begin{figure}
\includegraphics{img/dualascentstep.pdf}
\caption{My own illustration of the dual ascent step in the ADMM algorithm utilising dual decomposition based on \cite{GT12}.}
\end{figure}

\end{frame}

\note{
Method of Multipliers: is a way to make the algorithm more robust, (if in second line $\beta^t$ statt $\beta^{t+1}$)

alternating direction: semi-decomposable, i.e. keeping one variable fixed while updating the other

last step is called a dual variable update, this dual has nothing to do with two, but is connected to what is called a dual problem

dual variable update, we are working in the dual problem as "min L", thus convex problem, thus "dual decomposition" into subproblems which is possible by \cite{GT12} 

think of it as only the last line, sending $\mu$ to the updaters for $\beta$ and $\theta$

$\rho$ in last line can be thought of as "step length"

All notes on this slide: see the slides by \cite{boyd}
}

\begin{frame}
\frametitle{ADMM - Why?}
\begin{itemize}
\item[-] convex problems with nondifferentiable constraints
\item[-] blockwise computation
	\begin{itemize}
	\item[-] sample blocks
	\item[-] feature blocks
	\end{itemize}
\end{itemize}
\end{frame}

\note{
Details for blockwise computation in Exercise 5.12.
}

\begin{frame}
\frametitle{ADMM for the Lasso}
\framesubtitle{Problem}
Problem in Lagrangian form
\[\underset{\beta\in\R^p,\theta\in\R^p}{\text{minimize}}\left\{\frac{1}{2}\Norm{\mathbf{y}-\mathbf{X}\beta}_2^2+\lambda\Norm{\theta}_1\right\}\quad \text{such that}\ \beta-\theta=0 \]
\vspace{15pt}

Augmented Lagrangian
\[L_{\rho}(\beta,\theta,\mu):=\left\{\frac{1}{2}\Norm{\mathbf{y}-\mathbf{X}\beta}_2^2+\lambda\Norm{\theta}_1\right\}+\left\langle\mu,\beta-\theta\right\rangle+\frac{\rho}{2}||\beta-\theta||_2^2 \]
\end{frame}

\note{
In the problem, I can decompose into beta and theta terms, i.e.show $f(\beta)$ and $g(\theta)$

the problem itself and the constraints, 

A and B are unit matrices here

Computational cost: Initially $\mathcal{O}(p^3)$, which is a lot, for the SVD(singular value decomposition of $\mathbf{X}$), after that comparable to coordinate descent or composite gradient from earlier
}


\begin{frame}
\frametitle{ADMM for the Lasso}
\framesubtitle{Update}
Update
\begin{align*}
\beta^{t+1}&=(\mathbf{X}^T\mathbf{X}+\rho\mathbf{I})^{-1}(\mathbf{X}^T\mathbf{y}+\rho\theta^t-\mu^t)\\
\theta^{t+1}&=\mathcal{S}_{\lambda/\rho}(\beta^{t+1}+\mu^t/\rho)\\
\mu^{t+1}&=\mu^t+\rho(\beta^{t+1}-\theta^{t+1})
\end{align*}
where $\mathcal{S}_{\lambda/\rho}(z)=\text{sign}(z)(\norm{z}-\frac{\lambda}{\rho})_+$.

\end{frame}

\note{
$\mathcal{S}$ is a soft-thresholding parameter

Computational cost: Initially $\mathcal{O}(p^3)$, which is a lot, for the SVD(singular value decomposition of $\mathbf{X}$), after that comparable to coordinate descent or composite gradient from earlier
}


%------------------------------------------------
\section{Screening Rules}
%------------------------------------------------

\begin{frame}
\frametitle{Screening Rules}

\begin{itemize}
	\item very big data set, esp. huge number of predictors
	\item maybe too big to load into memory
	\item Screening rules eliminate predictors with minor calculation
	\item and very high / safe certainty (i.e. eliminated predictors would not show up in lasso model based on full data)
\end{itemize}

\note{Imagine a big data set, a very big data set, with such a huge design matrix, that you cannot load it into memory (RAM). Wh}




\end{frame}

\begin{frame}
\frametitle{What is a good predictor?}

includegraphics

correlation is covariance with some factors

covariance is an inner product on a vector space

high absolute correlation (=large absolute inner product) => high predictive power (see plots) => xj with largest inner product has predictive power, thus for that j we are most willing to accept some penalty from lambda


\end{frame}

\begin{frame}
\frametitle{SAFE Rules}


\end{frame}

\begin{frame}
\frametitle{Dual Polytope Projection (DPP)}
Suppose we want to calculate a lasso solution at $\lambda<\lambda_{\max}$. The DPP rule discards the $j^{th}$ variable if 
\[\norm{\mathbf{x}_j^T\mathbf{y}}<\lambda_{\max}-\Norm{\mathbf{x}_j}_2\Norm{\mathbf{y}}_2\frac{\lambda_{\max}-\lambda}{\lambda}\]

\vspace{5pt}
{\hspace{5pt}\Large Sequential DPP rule}
\vspace{15pt}

Suppose we have the lasso solution $\hat\beta(\lambda')$ at $\lambda'$ and want to screen variables for solutions at $\lambda<\lambda'$. We discard the $j^{th}$ variable if 
\[\norm{\mathbf{x}_j^T(\mathbf{y}-\mathbf{X}\hat{\beta}(\lambda'))}<\lambda'-\Norm{\mathbf{x}_j}_2\Norm{\mathbf{y}}_2\frac{\lambda_{\max}-\lambda}{\lambda}\]
\end{frame}

\begin{frame}
\frametitle{Global Strong Rule}
Suppose we want to calculate a lasso solution at $\lambda<\lambda_{\max}$. The global strong rule discards the $j^{th}$ variable if 
\[\norm{\mathbf{x}_j^T\mathbf{y}}<\lambda-(\lambda_{\max}-\lambda)=2\lambda-\lambda_{\max}\]

\vspace{5pt}
{\hspace{5pt}\Large Sequential Strong Rule}
\vspace{15pt}

Suppose we have the lasso solution $\hat\beta(\lambda')$ at $\lambda'$ and want to screen variables for solutions at $\lambda<\lambda'$. We discard the $j^{th}$ variable if 
\[\norm{\mathbf{x}_j^T(\mathbf{y}-\mathbf{X}\hat{\beta}(\lambda'))}<2\lambda-\lambda'\]
\end{frame}



%------------------------------------------------
\section{Minor-Max Algorithms}
%------------------------------------------------

\begin{frame}
\frametitle{Minorization-Maximization Algorithms (MMA)}
\begin{itemize}
\item[-] Problem: minimize $f(\beta)$ over $\beta\in\R^p$\\ for $f$ possibly non-convex
\item[-] Introduce additional variable $\theta$
\item[-] Use $\theta$ to majorize (bound from above) the objective function to be minimized
\end{itemize}


{\small Majorization-Minimization Algorithms work analoguosly.}
\end{frame}

\begin{frame}
\frametitle{MMA visually}
\begin{figure}
%\includegraphics[width=\textwidth]{img/minmaxalgo.png}
%\caption{Figure 5.10 from \cite{Has15}}
\includegraphics[height=150pt]{img/minmaxalgo2.png}
\caption{Figure from \cite{DL15}}
\end{figure}

\end{frame}

\begin{frame}
\frametitle{MMA analytically I}
Def. 
$\Psi:\R^p\times\R^p\to\R$ {\color{blue}majorizes} $f$ at $\beta\in\R^p$ if \[\forall\theta\in\R^p\quad \Psi(\beta,\theta)\ge f(\beta)\]
with equality for $\theta=\beta$.
\vspace{10pt}

Minor-Maxxalgorithm
\begin{itemize}
\item[-] initialize $\beta^0$
\item[-] update with $\beta^{t+1}=\argmin\limits_{\beta\in\R^p}\Psi(\beta,\beta^t)$
\end{itemize}
\end{frame}

\begin{frame}
\frametitle{MMA analytically II}
This scheme generates a sequence of $\beta$'s for which the cost $f(\beta^t)$ is nonincreasing, because
\[f(\beta^t)\stackrel{(i)}{=}\Psi(\beta^t,\beta^t)\stackrel{(ii)}{\ge}\Psi(\beta^{t+1},\beta^t)\stackrel{(iii)}{\ge} f(\beta^{t+1})\]

where 
\begin{itemize}
\setlength{\itemindent}{20pt}
\item[(i) \& (iii)] Definiton of majorize
\item[(ii)] $\beta^{t+1}$ is a minimizer of $\beta\mapsto\Psi(\beta,\beta^t)$
\end{itemize}
\end{frame}

\note{
for inequalities: show previous slide
}


%------------------------------------------------
\section{Alternating Minimizations}
%------------------------------------------------


\begin{frame}
\frametitle{Biconvexity}
Let's consider an example \dots

\vspace{5pt}
%\url{http://www.wolframalpha.com/input/?i=3D+plot+(1-xy)\%5E2,+x+in+\%5B-2,2\%5D,+y+in+\%5B-2,2\%5D}
{\color{blue}\href{http://www.wolframalpha.com/input/?i=3D+plot+(1-xy)\%5E2,+x+in+\%5B-2,2\%5D,+y+in+\%5B-2,2\%5D}{$f(\alpha,\beta)=(1-\alpha\beta)^2$}}

\note{
Mathematica: \texttt{3D plot (1-xy)\^{}2, x in [-2,2], y in [-2,2]}

The formula is a link.
}

\vspace{30pt}
\only<2>{Def. A function $f(\alpha,\beta):\R^m\times\R^n\to\R$ is {\color{gray}biconvex}, if for each $\alpha\in\R^m$ the function $\alpha\mapsto f(\alpha,\beta)$ is convex  and for each $\beta\in\R^n$ the function $\beta\mapsto f(\alpha,\beta)$ is convex.
\vspace{5pt}
Analoguosly, a set $\mathcal{C}\subseteq\mathcal{A}\times\mathcal{B}$, for $\mathcal{A, B}$ convex sets, is called \underline{biconvex}, if it is convex}
\end{frame}

\begin{frame}
\frametitle{Alternate Convex Search}

Block coordinate descent applied to $\alpha$ and $\beta$ blocks
\begin{enumerate}
\item Initialize $(\alpha^0,\beta^0)$ at some point in the biconvex set to minimize over
\item For $t=0,1,2,\dots$
	\begin{enumerate}[(i)]
	\item Fix $\beta=\beta^t$ and update $\alpha^{t+1}\in\argmin\limits_{\alpha\in\mathcal{C}_{\beta^t}}f(\alpha,\beta^t)$
	\item Fix $\alpha=\alpha^{t+1}$ and update $\beta^{t+1}\in\argmin\limits_{\alpha\in\mathcal{C}_{\alpha^{t+1}}}f(\alpha^{t+1},\beta)$
	\end{enumerate}
\end{enumerate}
\vspace{10pt}

For a function bounded from below, the algorithm converges to a partial optimum (i.e. as biconvexity, only optimal in one coordinate if the other coordinate is fixed).
\end{frame}

\begin{frame}[allowframebreaks]
\frametitle{References}
\footnotesize{
\begin{thebibliography}{99} % Beamer does not support BibTeX so references must be inserted manually as below
\bibitem[Hastie et al., 2015]{Has15} Trevor Hastie, Robert Tibshirani, and Martin Wainwright (2015)
\newblock Statistical learning with sparsity: the Lasso and
generalizations
\newblock \emph{CRC Press;} Boca Raton, FL%12(3), 45 -- 678.
\bibitem[de Leeuw, 2015]{DL15} Jan De Leeuw (2015)
\newblock Block Relaxation Methods in Statistics
\newblock \url{doi.org/10.13140/RG.2.1.3101.9607} (last accessed: 02.10.18)
\bibitem[Boyd]{boyd} S. Boyd 
\newblock Alternating Direction Method of Multipliers
\newblock \url{https://web.stanford.edu/~boyd/papers/pdf/admm_slides.pdf} (last accessed: 14.10.18)
\bibitem[Gordon and Tibshirani, 2012]{GT12} Geoff Gordon and Ryan Tibshirani (2012)
\newblock Uses of Duality
\newblock \url{https://www.cs.cmu.edu/~ggordon/10725-F12/slides/18-dual-uses.pdf} (last accessed: 14.10.18) 
\bibitem[Rubin, 2016]{ru16} Paul Rubin (2016)
\newblock What are the advantages of convex optimization compared to more general optimization problems?
\newblock \url{https://www.quora.com/What-are-the-advantages-of-convex-optimization-compared-to-more-general-optimization-problems} (last accessed: 14.10.18) 
\end{thebibliography}
}
\end{frame}
%Source: Trevor Hastie, Robert Tibshirani, and Martin Wainwright. Statistical learning with sparsity: the Lasso and generalizations. CRC Press, 2015.
%@unknown{unknown,
%author = {De Leeuw, Jan},
%year = {2015},
%month = {12},
%pages = {},
%title = {Block Relaxation Methods in Statistics - Part II}
%}

%------------------------------------------------
\begin{frame}
\Huge{\centerline{Comments \dots}}
\Huge{\centerline{Questions \dots}}
\Huge{\centerline{Suggestions \dots}}
\end{frame}


\begin{frame}
\Huge{\centerline{That's it.}}
\Huge{\centerline{Thanks for listening.}}
\vspace{20pt}
\large{\centerline{Fill out your feedback sheets!}}
%\Huge{\centerline{The End}}
\end{frame}

%----------------------------------------------------------------------------------------

\end{document} 